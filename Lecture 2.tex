% !TEX TS-program = pdflatex
% !TEX encoding = UTF-8 Unicode
%
% This is a simple template for a LaTeX document using the "article" class.
% See "book", "report", "letter" for other types of document.

\documentclass[11pt]{article} % use 12pt for true LSA-like documents


\usepackage[utf8]{inputenc} % set input encoding (not needed with XeLaTeX)

%%% Examples of Article customizations
% These packages are optional, depending whether you want the features they provide.
% See the LaTeX Companion or other references for full information.

%%% PAGE DIMENSIONS
\usepackage{geometry} % to change the page dimensions
%\geometry{a4paper} % or letterpaper (US) or a5paper or....
\geometry{margin=.75in} % for example, change the margins to 2 inches all round
% \geometry{landscape} % set up the page for landscape
%   read geometry.pdf for detailed page layout information

\usepackage{graphicx} % support the \includegraphics command and options

% \usepackage[parfill]{parskip} % Activate to begin paragraphs with an empty line rather than an indent

%%% PACKAGES
\usepackage{tipa} % !IPA font
\usepackage{covington} % better examples
\usepackage{hanging} % hanging indentation
\usepackage{qtree} % syntax trees
\usepackage[normalem]{ulem} % allows underlining
\usepackage{amssymb} % more math symbols
\usepackage{stmaryrd} % for making denotation brackets, etc.
\usepackage{setspace} % \singlespacing \doublespacing etc.
\usepackage{booktabs} % for much better looking tables
\usepackage{array} % for better arrays (eg matrices) in maths
\usepackage{paralist} % very flexible & customisable lists (eg. enumerate/itemize, etc.)
\usepackage{verbatim} % adds environment for commenting out blocks of text & for better verbatim
\usepackage{subfig} % make it possible to include more than one captioned figure/table in a single float
\usepackage{multirow} % span multiple rows/columns in tabular
\usepackage[section]{placeins} % prevent floats from going to awkward places
\usepackage{colortbl,varwidth,qtree,arydshln}

\usepackage{CJKutf8}

%%% HEADERS & FOOTERS
\usepackage{fancyhdr} % This should be set AFTER setting up the page geometry
\pagestyle{fancy} % options: empty , plain , fancy
\lhead{Lesson initially prepared by Blake H. Allen, LaTeX code and content modified by Kamila Kolpashnikova \\SJC Japanese Class 2015: Lesson 2}\chead{}\rhead{}
\lfoot{}\cfoot{}\rfoot{}

%%% SECTION TITLE APPEARANCE
\usepackage{sectsty}
\allsectionsfont{\sffamily\mdseries\upshape} % (See the fntguide.pdf for font help)
% (This matches ConTeXt defaults)

%%% ToC (table of contents) APPEARANCE
\usepackage[nottoc,notlof,notlot]{tocbibind} % Put the bibliography in the ToC
\usepackage[titles,subfigure]{tocloft} % Alter the style of the Table of Contents
\renewcommand{\cftsecfont}{\rmfamily\mdseries\upshape}
\renewcommand{\cftsecpagefont}{\rmfamily\mdseries\upshape} % No bold!

\newcommand \bal[2]{\only<#1>{\textbf}{#2}}
\newcommand{\bl}[2]{\bal{#1}{\alert<#1>{#2}}}

\newcommand{\subdot}{\textsubdot}
\newcommand{\up}{\textsuperscript}

\usepackage{hyperref}
\usepackage{xcolor}
\definecolor{dark-red}{rgb}{0.4,0.15,0.15}
\definecolor{dark-blue}{rgb}{0.15,0.15,0.4}
\definecolor{medium-blue}{rgb}{0,0,0.5}
\hypersetup{
    colorlinks, linkcolor={dark-red},
    citecolor={dark-blue}, urlcolor={medium-blue} }

\newcommand{\bex}[0]{\begin{example}}
\newcommand{\eex}[0]{\end{example}}
\newcommand{\q}[1]{\textipa{#1}}
\newcommand{\citey}{\citeyearpar}% uses natbib
\newcommand{\evalfun}[2][]{\ensuremath{\left\llbracket \mbox{#2} \right\rrbracket^{#1}}}
\newcommand{\lambt}[2]{\ensuremath{\lambda #1 \in D_{#2}}} %\lambt{x}{e}
\newcommand{\setof}[1]{\ensuremath{\left \{ #1 \right \}}}
\newcommand{\scell}[2][c]{\begin{tabular}[#1]{@{}c@{}}#2\end{tabular}} % \scell{} allows \\ in tabular
\newcommand{\ee}{\vspace{.10cm}\\} %ee for "Example End"

\linespread{1}

\setlength{\parindent}{0pt}

%%% END Article customizations

%%% The "real" document content comes below...

\begin{document}

\section{Set Phrases \& Pronunciation}

\begin{tabular}{l || l}

\textbf{Konnichiwa!} (Hello!) & Hiragana:\\
\textbf{Konbanwa.} (Good evening.) & \textbf{a ka sa ta} 
\begin{CJK}{UTF8}{min}あ・か・さ・た\end{CJK}\\
\textbf{Ohayoo gozaimasu.} (Good morning.) & \textbf{i ki shi chi} 
\begin{CJK}{UTF8}{min}い・き・し・ち\end{CJK}\\
\textbf{Arigatoo gozaimasu.} (Thank you.)  & \textbf{u ku su tsu} 
\begin{CJK}{UTF8}{min}う・く・す・つ\end{CJK}\\
\textbf{Sumimasen...} (Pardon me...)		& \textbf{e ke se te} 
\begin{CJK}{UTF8}{min}え・け・せ・て\end{CJK}\\
											& \textbf{o ko so to} 
\begin{CJK}{UTF8}{min}お・こ・そ・と\end{CJK}
\end{tabular}\\
- Sounds with doubled letters are pronounced longer: \textbf{e} vs. \textbf{ee}, \textbf{o} vs. \textbf{oo}, \textbf{n} vs. \textbf{nn}, etc.\\
- Also, note that [ee] is often written $<$\textbf{ei}$>$, and [oo] as $<$\textbf{ou}$>$.

\section{Nouns}
\begin{tabular}{l | l || l | l}
\textbf{watashi} & I						& \textbf{\textit{(name)}-san} & he/she\\
\textbf{anata} & you\\
\hline
\textbf{kore} & this (close to speaker)\\
\textbf{sore} & this/that (close to listener)\\
\textbf{are} & that (far from both)\\
\hline
\textbf{sensee} & teacher(s)				& \textbf{otoko (no hito)} & man\\
\textbf{gakusee} & student(s)				& \textbf{onna (no hito)}  & woman\\
\textbf{neko} & cat(s)						& \textbf{kuruma}		   & car\\
\textbf{inu} & dog(s)						& \textbf{ringo}		   & apple\\
\textbf{kutsu} & shoe(s)					& \textbf{raamen}		   & ramen\\
\textbf{purasuchikku} & plastic				& \textbf{paatii}		   & party\\
\hline
\textbf{dare} & who							& \textbf{itsu}				& when\\
\textbf{nan/nani} & what\\
\textbf{doko} & where\\
\end{tabular}\\
(Note: For most nouns, the singular and plural look/sound the same.)


\section{``is"}
$\bigstar$ \textit{A} is/am/are... \textit{B}. $\Rightarrow$ \textbf{\textit{A} wa \textit{B} desu.}
\begin{CJK}{UTF8}{min}(AはBです。)\end{CJK}\\
(Note: B is a noun that describes or is equivalent to A.)\ee

$\bigstar$ \textit{A} is/am/are... NOT \textit{B}. $\Rightarrow$ \textbf{\textit{A} wa \textit{B} ja arimasen.}
\begin{CJK}{UTF8}{min}(AはBじゃありません。)\end{CJK}\ee

$\bullet$ ``Kare wa Jaehyun desu."\\
$\bullet$ ``Kore wa tsukue desu."\\
$\bullet$ ``Blake wa onna no hito ja arimasen."\\
$\bullet$ ``Kiwa \textit{to} Jaehyun wa sensee desu."\\
(Note: \textbf{\textit{to}} = \textit{and}; it is used with nouns.)


\section{Questions \& Possession/Relationships}
\begin{tabular}{l || l}
$\bigstar$ (question sentence) $\Rightarrow$ \textbf{(sentence) ka} & $\bigstar$ \textit{A}'s \textit{B} $\Rightarrow$ \textbf{\textit{A} no \textit{B}}
\begin{CJK}{UTF8}{min}(AのB)\end{CJK}\\
&\\
$\bullet$ ``Jaehyun wa neko desu ka?" & $\bullet$ ``Anata wa dare no gakusee desu ka?"\\
$\bullet$ ``Kore wa isu desu ka?" & $\bullet$ ``Blake no kutsu wa doko desu ka?"\\
$\bullet$ ``Sensee wa dare desu ka?" & $\bullet$ ``Kono onna no hito wa dare desu ka?"\\
$\bullet$ ``Kore wa nan desu ka?" & $\bullet$ ``Watashi no isu wa purasuchikku no isu ja arimasen."

\end{tabular}


\section{Adjectives}

- Adjectives:\\
$\bigstar$1) modify (come before) nouns, or \\
$\bigstar$2) come before \textit{desu} to describe something.\ee
- They come in \textit{two types}: those ending in [-i] and those ending in [-na].
\begin{center}
\begin{tabular}{l l | l l}
\multicolumn{2}{c}{-i type}			 & \multicolumn{2}{c}{-na type}\\
\hline
\textbf{aoi} & blue 				& \textbf{kirei-na} & pretty	\\
\textbf{akai} & red 				& \textbf{kantan-na} & simple/easy \\
\textbf{utsukushii} & beautiful 	& \textbf{oshare-na} & fashionable \\
\textbf{oishii} & delicious 		& \textbf{hen-na}	& weird\\
\textbf{tanoshii} & fun 			& \textbf{iya-na}	& unpleasant\\
\textbf{atsui} & hot 				& \textbf{benri-na} & convenient\\
\textbf{samui} & cold 				& \textbf{genki-na} & healthy/feeling well\\
\textbf{muzukashii} & difficult 	& \textbf{anzen-na}	& safe\\
\textbf{ookii} & large 				& \textbf{kiken-na} & dangerous\\
\textbf{chiisai} & small \\
\textbf{atarashii}	& new \\
\textbf{warui} & bad \\
\end{tabular}
\end{center}

$\bullet$ \textbf{ookii neko} ``a large cat"\\
\begin{CJK}{UTF8}{min}おおきいねこです\end{CJK}\\
$\bullet$ \textbf{oshare-na kutsu} ``fashionable shoes"\ee

$\bullet$ \textbf{Anata wa utsukushii desu.} ``You are beautiful."\\
- Note that if you use a ``-na" adjective before ``desu", you leave out the ``na":\\
$\bullet$ \textbf{Genki desu ka?} ``Are you well? / How are you?"\\ (NOT \textit{genki-na desu ka})\ee

More examples:\\
$\bullet$ \textbf{Raamen wa oishii desu}.  ``Ramen is delicious."\\
$\bullet$ \textbf{Kore wa ookii tsukue desu.} ``This is a big desk." (cf. \textbf{Kono tsukue wa ookii desu.})\\
$\bullet$ \textbf{Warui hito wa kiken desu}.  ``Bad people are dangerous."\ee

\section{Negative Adjectives}

- Negation:(``not $\sim$")\\
- Form depends on the adjective type:\ee

$\bigstar$ not (adjective) $\Rightarrow$\\
-i $\rightarrow$ -kunai \\
-na $\rightarrow$ -ja nai \ee

$\bullet$ \textbf{oishikunai sushi} ``sushi that is not delicious" (literally: ``not-delicious sushi")\\
$\bullet$ \textbf{anzen ja nai isu} ``a chair that is not safe" ($\approx$ ``an unsafe chair")\\
$\bullet$ \textbf{Nihongo wa muzukashiku nai desu}. ``Japanese is not difficult."\\
$\bullet$ \textbf{Genki ja nai desu ka}?  ``Are you not feeling well?"\\

\section{Demonstratives}

\begin{tabular}{l l||l l | l l | l l | l l}
& & \multicolumn{2}{c}{`this' } & \multicolumn{2}{c}{`this/that'} & \multicolumn{2}{c}{`that'} & \multicolumn{2}{c}{(question)}\\
\hline
& & \multicolumn{2}{c}{\textbf{ko-}} & \multicolumn{2}{c}{\textbf{so-}} & \multicolumn{2}{c}{\textbf{a-}} & \multicolumn{2}{c}{\textbf{do-}}\\
\hline
noun & \textbf{-re}& \textbf{kore} & this (n.) & \textbf{sore} & this/that (n.) & \textbf{are} & that (n.) & \textbf{dore} & which one? \\
adjective & \textbf{-no}& \textbf{kono} & this (a.) & \textbf{sono} & this/that (a.) & \textbf{ano} & that (a.) & \textbf{dono} & which? \\
place & \textbf{-ko}& \textbf{koko} & here & \textbf{soko} & there & \textbf{\textbf{asoko}} & over there & \textbf{dono} & where? \\
\end{tabular}\\\vspace{1cm}

$\bullet$ \textbf{Kore wa isu desu}.  ``This is a chair."\\
$\bullet$ \textbf{Kono neko wa kawaii desu.} ``This cat is cute."\\
$\bullet$ \textbf{Koko no raamen wa oishii desu}. ``The ramen here (=of this place) is delicious."\ee

$\bullet$ \textbf{Sono hito wa dare desu ka}?  ``Who is that person?"\ee

$\bullet$ \textbf{Ano inu wa kiken ja nai}.  ``That dog (someone else's) is not dangerous."\ee

$\bullet$ \textbf{Anata no isu wa dore desu ka}?  ``Which one is your chair?"\\


\section{Sentence Practice!}

Work with a partner (or two) and try to translate these sentences to/from Japanese:\ee

$\bullet$ This desk is large.\\
$\bullet$ The food here (literally: food of here) is delicious.\\
$\bullet$ Small dogs are fashionable.\\
$\bullet$ That is not safe.\\
$\bullet$ Which is the new car? (literally: the new car is which one?)\ee

$\bullet$ Sensee wa hen ja nai.\\
$\bullet$ Akai ringo wa oishii desu.\\
$\bullet$ Sono atarashikunai kuruma wa dare no kuruma desu ka?\\
$\bullet$ Asoko no ``hen na inu" wa inu ja nai.  Watashi no neko desu!! 


\end{document}
